%%%%%%%%%%%%%%%%%%%%%%%%%%% asme2ej.tex %%%%%%%%%%%%%%%%%%%%%%%%%%%%%%%

%%%%%%%%%%%%%%%%%%%%%%%%%%%%%%%%%%%%%%%%%%%%%%%%%%%%%%%%%%%%%%%%%%%%%%

%%% use twocolumn and 10pt options with the asme2ej format
\documentclass[twocolumn,10pt]{asme2ej}
%
\usepackage{epsfig} %% for loading postscript figures

%% The class has several options
%  onecolumn/twocolumn - format for one or two columns per page
%  10pt/11pt/12pt - use 10, 11, or 12 point font
%  oneside/twoside - format for oneside/twosided printing
%  final/draft - format for final/draft copy
%  cleanfoot - take out copyright info in footer leave page number
%  cleanhead - take out the conference banner on the title page
%  titlepage/notitlepage - put in titlepage or leave out titlepage
%
%% The default is oneside, onecolumn, 10pt, final


\title{MATH 371 Fall Quarter MCM 2019 Problem B}


\begin{document}

\maketitle


\begin{abstract}
{\it This is the abstract. This is the thing we're going to spend a LOT of time writing, it's very important. We still need to write the abstract and the introduction
}
\end{abstract}

%\begin{nomenclature}
%\entry{A}{You may include nomenclature here.}
%\entry{$\alpha$}{There are two arguments for each entry of the nomemclature environment, the symbol and the definition.}
%\end{nomenclature}

%The primary text heading is  boldface and flushed left with the left margin.  The spacing between the  text and the %heading is two line spaces.

\section{Introduction}

This is the introduction

\section{Ideation}
We identified two significant problems to focus on in our model: packaging the drones and medi-packs to be deployed in the disaster area, and routing the drones to deliver necessary supplies according to the requirements from the delivery locations in attachment 4. We generalized a solution early on that involved algorithmically optimizing the packing space in each cargo container and modeling many routes to choose an ideal route based on a cost function, thereby eliminating two degrees of freedom. The remaining freedom is in choosing which drone type to use, which we determined could be implemented in the flight-path algorithm.

In this problem, there is already a lot of information established: the dimensions of all interacting products are fully defined, the performance capabilities of the drones are known, and the exact locations of delivery locations in the form of latitude and longitude points are provided. We used all this information to our advantage to narrow the scope of the problem and allow us more time to model the specific operational flow of our disaster response system. Despite this, however, several assumptions and simplifications were made in our model to reduce complexity which are outlined below.

\section{Assumptions}

\begin{itemize}
	\item Scheduling
	\begin{itemize}
		\item[--] Only the minimum supplies required daily are delivered to delivery locations, i.e. no rotating schedules and stockpiling of supplies to avoid needing to deliver to each daily
		\item[--] Additional time for operations is not considered: no time is needed to be spent in between flights for loading supplies or recharging the drones, and no time is needed to land and deliver supplies at delivery locations
	\end{itemize}
    \item Packing
    \begin{itemize}
    	\item[--] Elements can be stacked in any configuration without structural limitations
    \end{itemize}
	\item Flight path routing
	\begin{itemize}
		\item[--] Paths are perfectly straight
		\item[--] Every path only has either the delivery location or storage container as origin and destination
		\item[--] Paths are modeled in two dimensions i.e. no altitude changes are considered
	\end{itemize}
	\item Environmental effects
	\begin{itemize}
		\item[--] Influences from wind are neglected
		\item[--] The drones are assumed to be unobstructed by terrain
		\item[--] The drones do not experience any malfunctions
		\item[--] The earth’s curve is neglected
	\end{itemize}
\end{itemize}


\section{Overview of Algorithms}
 
 
\subsection{Flight Path Algorithm}
The flight path algorithm inputs the coordinates of each delivery location and its required medi-packs, information about each drone type, and assumes the starting location to be (0,0). It generates a list of all possible configurations of flight paths, each one represented as a vector of 2D lines. Any configurations that do not visit required hospitals , otherwise fail to meet delivery requirements, or cannot be sufficiently packed into the storage container(s) are eliminated from the list. A cost function that divides the total number of packages delivered by total distance over maximum speed is applied to the remaining list of configurations to find the least expensive flight paths. The algorithm is written in MATLAB and outputs a graph which represents the flight paths as lines and delivery locations as points, and the program outputs the total number of optimized flight paths found in the console.

The algorithm does not apply any calculation to determine how many medi-packs can fit in each drone configuration: analysis was done by hand to evaluate this for each drone type, and the maximum number of medi-packs that each drone can carry is hardcoded into the algorithm. 

Many limitations from this algorithm become immediately apparent: the real-life terrain that would be faced by the drones is ignored in favor of an ultra-simple “by the crow flies” model of flight. By utilizing this approach, the model also invalidates any capability of maximizing surveillance of roadways on the island of Puerto Rico, which was identified as a major focus for the DroneGo disaster response system.

\subsection{Packing Algorithm}
A packing algorithm is used to determine whether the materials necessitated by each flight configuration can be spatially packed into the dimensions of a standard ISO dry cargo container. The algorithm inputs the dimensions of each drone type and each medipack, and outputs a simple boolean expressing whether the configuration will fit or not. The algorithm  inputs the items into the cargo area by creating columns with base dimensions defined by the packages, and checks whether each new package can be stacked in an existing column before resorting to making a new column. Each package is only considered for one optimal orientation which is hardcoded prior. If the algorithm still has packages left to pack, but cannot create any more columns, it will return a 0.


\end{document}
